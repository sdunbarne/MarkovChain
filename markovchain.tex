%%% -*-LaTeX-*-
%%% markovchain.tex.orig
%%% Prettyprinted by texpretty lex version 0.02 [21-May-2001]
%%% on Fri Jun 12 08:02:20 2020
%%% for Steve Dunbar (sdunbar@family-desktop)

\documentclass[12pt]{article}

\input{../../../../etc/macros}
\input{../../../../etc/mzlatex_macros}
%% \input{../../../../etc/pdf_macros}

\bibliographystyle{plain}

\begin{document}

\myheader \mytitle

\hr

\sectiontitle{Markov Chains}

\hr

\usefirefox

% \hr
% \visual{Study Tip}{../../../../CommonInformation/Lessons/studytip.png}
% \section*{Study Tip}

\hr

\visual{Rating}{../../../../CommonInformation/Lessons/rating.png}
\section*{Rating} %one of
% Everyone: contains no mathematics.
% Student: contains scenes of mild algebra or calculus that may require guidance.
Mathematically Mature:  may contain mathematics beyond calculus with
proofs.  % Mathematicians Only: prolonged scenes of intense rigor.

\hr

\visual{Section Starter Question}{../../../../CommonInformation/Lessons/question_mark.png}
\section*{Section Starter Question}

Consider the magical Land of Oz where the weather follows a pattern.  If
it is raining today, then there is a \( 50\% \) chance of raining again
tomorrow, and a \( 25\% \) chance of either having a nice day or a snowy
day tomorrow.  Similarly if it is snowing today, there is a \( 50\% \)
chance of having snow tomorrow, and a \( 25\% \) chance of either having
a nice day, or a rainy day tomorrow.  Also the land of Oz never has two
nice days in a row, with equal chance of rain or snow the day after a
nice day.  What is the chance it is nice in two days given it is rainy
today?

\hr

\visual{Key Concepts}{../../../../CommonInformation/Lessons/keyconcepts.png}
\section*{Key Concepts}

\begin{enumerate}
    \item
        A \defn{stochastic process} is, in the simplest terms, a random
        function.  A specific outcome of the process must be specified
        in terms of the probability of it happening.
    \item
        A discrete time discrete state space stochastic process is a
        \defn{Markov chain} if \( \Prob{X_{n+1} \given X_n, X_{n-1},
        \dots, X_1} = \Prob{X_{n+1} \given X_n} \).  That is, the
        conditional probability depends only on the immediately
        preceding state, not the history.
    \item
        Denote the probability of making the transition from state \( x_i
        \) to \( x_j \) at time \( n \) as:
        \[
            p_{ij}(n)=\Prob{O_{n+1}=x_j \given O_{n}=x_i}.
        \] If the probability of a state transition in a Markov process
        does not depend on \( n \), then it is a \defn{time-homogeneous
        Markov chain}.  If the state space is finite, say \( \card{\mathcal
        {X}} = k \), then the \( p_{ij} \) are the entries of an \( k
        \times k \) matrix \( P \) called the \defn{transition
        probability matrix}.
    \item
        The transition probability for multiple time steps is the matrix
        \( P^n \), and this is known as (the finite Markov Chain version
        of) the \textit{Chapman-Kolmogorov Theorem}.
    \item
        The Fundamental Theorem of Markov Chains says:  Let \( \mathcal{X}
        \) be a finite state space and let \( P_{ij} \) define a Markov
        chain on \( x_i \).  If there is an integer \( n_0 \) such that \(
        (P^{n_0})_{ij} > 0 \) for all \( n > n_0 \), then \( P \) has a
        unique stationary distribution \( \mathbf{\pi} \), and as \( n
        \to \infty \), \( (P^n)_{ij} \to \pi_{j} \) for each \( i,j \).
    \item
        \defn{Absorbing states} are states that once entered are never
        left.
    \item
        The limiting matrix can be found by putting the matrix in \defn{canonical
        form}, then finding the \defn{fundamental matrix}, and then
        multiplying blocks from the canonical matrix.
\end{enumerate}

\hr

\visual{Vocabulary}{../../../../CommonInformation/Lessons/vocabulary.png}
\section*{Vocabulary}
\begin{enumerate}
    \item
        A \defn{stochastic process} is a random function on a domain.
    \item
        The outcome of a \defn{discrete time discrete state space
        process} is one of several states at a sequence of times.  Along
        a specific sample, as time progresses the observed sequence of
        states from the set \( \mathcal{X} \) is a \defn {sample path}.
    \item
        A \defn{discrete time Markov process} is a stochastic process
        having a fixed probability that it will be in state \( x_j \) at
        time \( n+1 \) when the process is in state \( x_i \) at time \(
        n \), with no dependence on the prior history.
    \item
        If the probability of a state transition in a Markov process
        does not depend on \( n \), it is a \defn{time-homogeneous
        Markov chain}.
    \item
        If the values in each the rows of a matrix sum to 1, it is a
        \defn{stochastic matrix}.
    \item
        A \defn{Markov chain} is a discrete time discrete state space
        defined by its transition probability matrix \( P \) with
        entries
        \[
            p_{ij}(n)=\Prob{X_{n+1}=x_j \given X_n=x_i}.
        \] The entries must satisfy \( P_{ij} \ge 0 \) and \( \sum_ {j=1}^k
        P_{ij} = 1 \) for each \( i=1, \dots, k \), so that each row is
        a probability distribution on \( X \). % \item
        %     The sequence of states \( X_0 = x_i, X_1 = x_j, X_2 = x_l
        %     \dots \) is a \defn{sample path} of the Markov chain.
        % \item
        %     A chain is \defn{ergodic}, also called \defn{irreducible}, if it
        %     is possible to move from any one state to any other, not
        %     necessarily in one move.
        % \item
        %     A chain is \defn{regular} if some power of the transition matrix
        %     has all positive entries.
        % \item
        %     A regular Markov chain has a \defn{stationary distribution} \(
        %     \pi_i > 0 \), and \( \sum_{i=1}^k \pi_i = 1 \) with \(
        %     \mathbf{\pi} \) satisfying
        %     \[
        %         \sum\limits_{i=1}^k \pi_i P_{ij} = \pi_{j}.
        %     \] Thus \( \mathbf{\pi} \) is a left eigenvector of \( P \) with
        %     eigenvalue \( 1 \).  Alternative terminology is that the Markov
        %     chain is \defn{stable on the distribution \( \mathbf{\pi} \)}
        %     and that \( \pi \) is the \defn{stable distribution} for the
        %     Markov chain.
        % \item
        %     A Markov chain is \defn{aperiodic} if for all states \( i \) and
        %     \( j \),
        %     \[
        %         \gcd\setof{n}{(P^n)_{i,j} > 0} = 1.
        %     \]
        % \item
        %     State \( j \) is said to be \defn{accessible} from state \( j \)
        %     if for some integer \( n \ge 0 \), \( P_{ij}^n > 0 \), that is
        %     there is a positive probability that in a finite number of
        %     steps, state \( j \) can be reached from state \( i \).  Two
        %     states \( i \) and \( j \), each of which is accessible from the
        %     other, are said to be \defn{communicate}.  The property of
        %     communicating is an equivalence relation, dividing the set of
        %     states into equivalence classes.
        % \item
        %     A state is said to be \defn{recurrent} if and only if starting
        %     from state \( i \) the probability of returning to state \( i \)
        %     after a finite number of steps is \( 1 \).  A state that is not
        %     recurrent is \defn{transient}.  A state \( i \) is recurrent if
        %     and only if \( \sum_{i=1}^{\infty} P_{ii}^n = \infty \).
    \item
        States which once entered are never left are \defn{absorbing
        states}.
    \item
        Given a Markov chain defined by transition matrix \( P \) with \(
        u \) absorbing states and \( v \) non-absorbing states where \(
        u + v = k \), then the matrix has the \defn{transition
        probability matrix canonical form}
        \[
            P=
            \begin{pmatrix}
                I & 0 \\
                R & Q \\
            \end{pmatrix}
        \] where \( I \) is the \( u\times u \) identity matrix, and \(
        0 \) is the \( u\times v \) matrix of zeros.  The matrix \( N=(I-Q)^
        {-1} \) is called the \defn{fundamental matrix}.
\end{enumerate}

\hr

\section*{Notation}
\begin{enumerate}
    \item \( \mathcal{X}=\{x_1,x_2,\ldots,x_i, \ldots, \} \) -- state
      space of a discrete space stochastic process
    \item \( \card{\mathcal{X}} = k \) -- cardinality of a finite
      state space
    \item \( P \) -- \( k times k \) transition probability matrix
    \item \( \nu \) -- A dummy variable for summation
    \item \( (P^n)_{ij} \) -- the \( i,j \) entry of the \( n \)th
      power of \( P \).
    \item \( \mathbf{\pi} \) -- stationary distribution satisfying
\[
    \sum\limits_{\nu=1}^k \pi_\nu P_{\nu j} = \pi_j}
\] with \( \pi_\nu > 0 \), and \( \sum_{\nu=1}^k \pi_\nu = 1 \)
\item \[
    P=
    \begin{pmatrix}
        I & 0 \\
        A & T \\
    \end{pmatrix}
\] -- canonical form for the transition matrix \( P \) with \( a \)
absorbing states and \( t \) transient states where \( a + t = k \),
where \( I \) is the \( a\times a \) identity matrix, and \( 0 \) is
the \( a\times t \) matrix of zeros.
\end{enumerate}
    
\visual{Mathematical Ideas}{../../../../CommonInformation/Lessons/mathematicalideas.png}
\section*{Mathematical Ideas}

\subsection*{Introduction}

This section is a survey of basic concepts and facts about Markov
processes and Markov chains.  Detailed treatments are in
\cite{durrett09, grinstead97, karlin75, kemeny74, kemeny60, levin09,
ross06, stirzaker05-stoch-proces-model}, among other standard
references.

Begin by understanding where Markov chains fit in the larger landscape
of probability theory.

\begin{definition}
    A \defn{stochastic process} is%
    \index{stochastic process}
    a random function on a domain.  The domain can be discrete or
    continuous.  The domain is often interpreted as time.  Stochastic processes
    are in contrast to deterministic processes.  A specific outcome of
    the process at some time must be specified in terms of the
    probability of it happening.
\end{definition}

\begin{example}
Consider the following simple example:
\begin{align*}
    f(t) &= 2t \\
    g(t) &=
    \begin{cases}
        2t & \text{ with probability \( 0.98 \) }\\
        3t & \text{ with probability \( 0.02 \)}.
    \end{cases}
\end{align*}
For \( f(2) \) the outcome will always be 4, it is deterministic.
However for \( g(2) \) the most likely outcome is 4 but this will fail
to be true for about \( 2\% \) of the trials.
\end{example}

\begin{definition}
    The outcome of a \defn{discrete time discrete state space stochastic
    process} is one of several discrete states at a sequence of
  discrete times.%
    \index{discrete time discrete state
        space stochastic process}
    For the purposes of this introduction call these states \( \mathcal{X}=\{x_1,x_2,\ldots,x_i,
    \ldots, \} \).  Sometimes the discrete state space \( \mathcal{X} \) is
    infinite, but often the state space is finite.  At time \( n \),
    guided by some information, the process acts in a probabilistic
    manner and outputs some \( x_i \).  Thus along a specific sample, as
    time progresses an observed sequence of states from the set \(
    \mathcal{X} \) is a \defn{sample path}.
\end{definition}

\begin{definition}
    A \defn{discrete time Markov process} \( X_n \) is a stochastic process having
    a fixed probability that it will be in state \( x_j \) at time \(
    n+1 \) when the process is in state \( x_i \) at time \( n \), with
    no dependence on the prior history.%
    \index{discrete time Markov process}
    Symbolically,
    \begin{multline*}
        \Prob{X_{t_{n+1}} \given X_{t_{n}} = x_n, X_{t_{n-1}} = x_{n-1},
        \dots X_{t_1} = x_{1}, X_{t_0} = x_{0}} = \\
        \Prob{X_{t_{n+1}} \given X_{t_{n}} = x_{n}}
    \end{multline*}
    Denote the probability of making the transition from state \( x_i \)
    to \( x_j \) at time \( n \) as:
    \[
        P_{ij}(n)=\Prob{x_{n+1}=x_j \given X_n=x_i}.
    \]
\end{definition}

\begin{example}
  As a contrast, consider a discrete time stochastic process
  modeling the daily closing price of a
company's stock.  At time \( n \), there is a wealth of available
information influencing the current price:  the past prices of the stock, the current market
conditions, the weather forecast for the next few days, even the price
of tea in China. Instead, modeling the daily closing price of a stock as
a discrete time \emph{Markov process}, at time \( n + 1 \) the \emph{only}
information that would affect the process would be the observed stock
price at time \( n \).
\end{example}

\begin{definition}
    If the probability of a state transition in a Markov process does
    not depend on \( n \), then it is a \defn{time-homogeneous Markov
    chain}.%
    \index{time-homogeneous Markov chain}
    This means \( P_{ij}(n)=P_{ij}(m) \) for all \( n,m \). That is, the
    transition probabilities from \( x_i \) to \( x_j \) are time
    invariant with fixed \( P_{ij} \), called the transition
    probabilities from \( x_i \) to \( x_j \).  If the state space is
    finite, say \( \card{\mathcal{X}} = k \), then the \( P_{ij} \) are
    the entries of an \( k \times k \) matrix \( P \) called the
    transition probability matrix.
\end{definition}

\subsection*{Example of a Markov Chain}

The following is a classic example from
\cite{kemeny74}, repeated in
\cite{grinstead97} and many other locations.  The example puts the
concepts in context as well as introducing some other ideas.

\begin{example}
    Consider the magical Land of Oz where the weather follows a pattern.
    If it is raining today, then there is a \( 50\% \) chance of raining
    again tomorrow, and a \( 25\% \) chance of either having a nice day
    or a snowy day tomorrow.  Similarly if it is snowing, there is a \(
    50\% \) chance of again having snow, and a \( 25\% \) chance of
    either having a nice day, or a rainy day.  The land of Oz never has
    two nice days in a row, and Oz equally has rain and snow the day
    after a nice day.

    First label the states conveniently as \( R, N, S \), instead of \(
    x_1, x_2, x_3 \).  With the information above, define \( P_{RR} =
    1/2 \), \( P_{RN}=1/4 \) and so on.  Writing out every possible
    transition gives \( 3^2 \) distinct possibilities. With \( k \)
    states, the process has \( k^2 \) transitions with the probability \(
    P_{ij} \) of going from \( x_i \) to \( x_j \).

    Use matrices to represent the transition probabilities for the
    Markov chain.  In the example, the transition matrix is (the row and
    column labels of \( R \), \( N \) and \( S \) are for convenience
    only and are not usually included in the display of the matrix)
    \[
        P = \bordermatrix{ & R & N & S \cr
        R & 1/2 & 1/4 & 1/4 \cr
        N & 1/2 & 0 & 1/2 \cr
        S & 1/4 & 1/4 & 1/2 }.
    \] Note that all the rows sum to 1, thus this is a \defn{stochastic
    matrix}.

    How to calculate the chance it is nice in two days given it is rainy
    today?  Use the Law of Total Probability to calculate this chance.
    This considers every possible weather combination with their
    respective conditional probabilities.  See Figure~%
    \ref{fig:markovchains:twostepprob} for a diagram of the combination
    process.  Mixing the \( R=x_1 \), \( N = x_2 \), \( S = x_3 \)
    notation, this example has
    \[
        \Pr[ X_{n+2} = N | X_{n} = R] = \sum_{\nu=1}^3 P_{1 \nu}P_{\nu 2}.
    \]
    \begin{figure}
        \centering
\begin{asy}
            size(5inches);

            real myfontsize = 12; real mylineskip = 1.2*myfontsize; pen
            mypen = fontsize(myfontsize, mylineskip); defaultpen(mypen);

            real r = 0.10;

            path cR0 = circle( (0,1), r); pair cR0a = relpoint(cR0, 0);
            draw(cR0); label("R", (0,1));

            path cR1 = circle( (1,1), r); pair cR1a = relpoint(cR1, 0);
            pair cR1z = relpoint(cR1, 0.5); draw(cR1); label("R", (1, 1));
            path cN1 = circle( (1,0), r); pair cN1a = relpoint(cN1, 0);
            pair cN1z = relpoint(cN1, 0.5); draw(cN1); label("N", (1, 0));
            path cS1 = circle( (1,-1), r); pair cS1a = relpoint(cS1, 0);
            pair cS1z = relpoint(cS1, 0.5); draw(cS1); label("S", (1, -1));

            draw("\( 1/2 \)", cR0a -- cR1z, Arrow); draw("\( 1/4 \)",
            cR0a -- cN1z, Arrow); draw("\( 1/4 \)", cR0a -- cS1z, Arrow);

            path cN2 = circle( (2,0), r); pair cN2z = relpoint(cN2, 0.5);
            draw(cN2); label("N", (2,0));

            draw("1/4", cR1a -- cN2z, Arrow); draw("0", cN1a -- cN2z,
            Arrow); draw("1/4", cS1a -- cN2z, Arrow);
\end{asy}
        \caption{A schematic diagram of the calculation of the two-step
        transition probability.}%
        \label{fig:markovchains:twostepprob}
    \end{figure}
    For two days, this is not too tedious, but what about the
    probability it will be nice a week from today given it is rainy
    today?

    Now matrix notation is convenient.  Notice that the probability
    above of going from \( R \) to \( N \) two days hence is the \( (1,
    2) \) entry in the matrix product of \( P \) with itself.  More
    generally, the transition probability through \( n \) steps is the
    matrix \( P^n \), and this is known as (the finite Markov Chain
    version of) the \textit{Chapman-Kolmogorov Theorem}.  Thus%
    \index{Chapman-Kolmogorov Theorem}
    consider the following matrices from the example:
    \[
        P^2 = \bordermatrix{ & R & N & S \cr
        R & 7/16 & 3/16 & 6/16 \cr
        N & 6/16 & 4/16 & 6/16 \cr
        S & 6/16 & 3/16 & 7/16 }
    \] and
    \[
        P^7 = \bordermatrix{ & R & N & S \cr
        R & 0.400024 & 0.200012 & 0.399963 \cr
        N & 0.400024 & 0.199951 & 0.400024 \cr
        S & 0.399963 & 0.200012 & 0.400024 }.
    \] Hence if it is raining today, then the chance it is nice in $2$ and
    $7$ days is \( 3/16 \) and \( 0.200012 \) respectively.
\end{example}

\subsection*{Formal Definitions}

Let the state space be a finite set \( \mathcal{X}=\{x_1,x_2,\ldots,x_k\}
\).
\begin{definition}
    A \defn{Markov chain} is a discrete time discrete state space
    stochastic process%
    \index{Markov chain}
    defined by its transition probability matrix \( P \) with entries
    \[
        P_{ij}(n)=\Prob{X_n=x_j \given X_{n-1}=x_i}.
    \] The entries must satisfy \( P_{ij} \ge 0 \) and \( \sum_{\nu=1}^k P_
    {i \nu} = 1 \) for each \( i=1, \dots, k \), so that each row is a
    probability distribution on \( X \).
\end{definition}

This means that the Markov chain is a random walk among the states with
dependence only on the current state.  If \( x_i \) is the current
state, choose \( x_j \) with probability \( P_{ij} \); from \( x_j \)
choose \( x_l \) with probability \( P_{jl} \), and so on.  The sequence
of outcomes \( X_0 = x_i, X_1 = x_j, X_2 = x_l \dots \) is a \defn{sample
path} of the Markov chain.%
\index{sample path}

In terms of conditional probabilities,
\begin{align*}
    \Prob{X_1 = x_j \given X_0 = x_{i}} &= P_{ij} \\
    \Prob{X_2 = x_l, X_1 = x_j \given X_0 = x_i} &= P_{ij}P_{jl}.  \\
\end{align*}
Then combining over all the intermediate possibilities
\[
    \Prob{X_2 = x_l \given X_0 = x_i} = \sum\limits_{j =1}^k P_{ij}P_{jl}
\] which is matrix multiplication.  This is the \textit{Chapman-Kolmogorov
equation} for finite Markov chains.%
\index{Chapman-Kolmogorov equation}
Extending this observation by induction
\[
    \Prob{X_n = j \given X_0 = i} = (P^n)_{ij}
\] the \( i,j \) entry of the \( n \)th power of \( P \).

A Markov chain has a \defn{stationary distribution} if there is a vector
\( \mathbf{\pi} \) satisfying
\[
    \sum\limits_{i=1}^k \pi_i P_{ij} = \pi_{j}
\] with \( \pi_i > 0 \), and \( \sum_{i=1}^k \pi_i = 1 \).  Thus \(
\mathbf{\pi} \) is a left eigenvector of \( P \) with eigenvalue \( 1
\).  The Perron Theorem guarantees that such a left eigenvector exists
for probability transition matrices.
Alternative terminology is that the Markov chain is \defn{stable on the
distribution \( \mathbf{\pi} \)} and that \( \pi \) is the \defn{stable
distribution}%
\index{stable distribution}
for the Markov chain.  The probabilistic interpretation of the left
eigenvector equation is ``pick \( x_i \) from \( \mathcal{X} \) with
probability \( \pi_i \) and take a step with probability \( P_{ij} \),
the probability of being at \( x_j \) is \( \pi_{j} \)''.  Thus \(
\mathbf{\pi} \) is stationary for the evolution of the Markov chain.

The precise definitions of the conditions on a Markov chain for the
following Fundamental Theorem to hold are in the next section ``Classes
of States and Stationary Distributions''.  The proof of the Fundamental
Theorem is also in the next section ``Classes of States and Stationary
Distributions''.
\begin{theorem}[Fundamental Theorem of Markov Chains]
    For an irreducible, positive recurrent and aperiodic Markov chain \(
    \lim_{n \to \infty} (P^n)_{ij} \) exists and is independent of \( i \).
    Furthermore, letting
    \[
        \pi_j = \lim_{n \to \infty} (P^n)_{ij}
    \] then \( \pi_j \) is the unique non-negative solution of
    \begin{align*}
        \sum\limits_{i} \pi_{i} P_{ij} &= \pi_{j},\\
        \sum\limits_{i} \pi_{i} &= 1.%
    \end{align*}
\end{theorem}
\index{Fundamental Theorem of Markov Chains}

An irreducible, positive recurrent and aperiodic Markov chain with
transition matrix \( P \) has the following property:  As \( n\to\infty \),
the matrix \( P^n \) approaches a limiting matrix \( P^\infty \), where \(
P^\infty \) consists of \( n \) copies of the same row vector.  This
matrix has different names in different texts, two common names are the
``limit matrix'', or the matrix of ``limiting probabilities.''%
\index{limit matrix}
\index{limiting probabilities}

The probabilistic content of the Fundamental Theorem is that from any
starting state \( x_i \), for large \( n \) the \( n \)th step of a
sample path of the Markov chain has a probability close to \( \pi_j \)
of being at \( x_j \) if \( n \) is large.  Later sections address the
question of what ``probability close to \( \pi_j \)'' means and how
large \( n \) must be for this probability relationship to hold.

\begin{example}
    For the weather in Oz example, consider \( P^7 \) further:
    \[
        P^7 =
        \begin{pmatrix}
            & R & N & S \\
            R & 0.400024 & 0.200012 & 0.399963 \\
            N & 0.400024 & 0.199951 & 0.400024 \\
            S & 0.399963 & 0.200012 & 0.400024
        \end{pmatrix}
        .
    \] Note that the chance of being in any of the three states after
    seven days is almost independent of the initial state.  This means
    the weather in seven days is almost completely independent of what
    it is today. Considering \( P^n \) for larger values of \( n \), the
    row vectors approach \( \mathbf{\pi}=(2/5,1/5,2/5) \).
\end{example}

\begin{example}
  Convergance to a stationary distribution works for the weather
  example, but it will not work for every matrix.
Consider:
\[
    Q =
    \begin{pmatrix}
        0 & 1 \\
        1 & 0
    \end{pmatrix}
    .
\] This matrix describes a two-state system switching states at each
time step.  Thus \( Q^2 = I \), \( Q^3=Q \), \( Q^4=I \) and so on.
This particular matrix does not have the property of begin aperiodic.
\end{example}

% A Markov chain is \defn{ergodic}, also called \defn{irreducible},%
% \index{ergodic}
% \index{irreducible}
% if it is possible to move from any one state to any other, not
% necessarily in one move.  Using the powers of probability transition
% matrix, this says that for all states \( i \) and \( j \), there exists \(
% n_0 \) such that \( (P^{n_0})_{ij} \ne 0 \).  Thus both \( P \) and \( Q
% \) represent ergodic chains.  A chain is \defn{regular}%
% \index{regular}
% if some power of the transition matrix has all positive entries.  While \(
% P \) is such that \( P_{22}=0 \), we see that \( P^2 \) has all nonzero
% entries, thus \( P \) is a regular chain.  But \( Q^n \) will always
% have zeros, so \( Q \) is not a regular chain.  In general every regular
% chain is ergodic, but not every ergodic chain is regular.

% A Markov chain is \defn{aperiodic} if for all states \( i \) and \( j \),%
% \index{aperiodic}
% \[
%     \gcd\setof{n}{(P^n)_{ij} > 0} = 1.
% \]

% \subsection*{Classification of States}
%Now we consider what the probability matrix represents in a
%different context.  Consider the following graph:
%\begin{center}
%\includegraphics[scale=.5]{oz_graph.eps}
%\end{center}

% State \( j \) is said to be \defn{accessible} from state \( i \) if for
% some integer \( n \ge 0 \), \( P_{ij}^n > 0 \), that is there is a
% positive probability that in a finite number of steps, state \( j \) can
% be reached from state \( i \).  Two states \( i \) and \( j \), each of
% which is accessible from the other, are said to \defn{communicate}.  The
% property of%
% \index{communicate}
% communicating is an equivalence relation, dividing the set of states
% into equivalence classes.  A state is said to be \defn{recurrent} if%
% \index{recurrent}
% and only if starting from state \( i \) the probability of returning to
% state \( i \) after a finite number of steps is \( 1 \).  A state that
% is not recurrent is \defn{transient}.  A theorem characterizing
% recurrence is%
% \index{transient}
% the following:
% \begin{theorem}
%     A state \( i \) is recurrent if and only if \( \sum_{i=1}^{\infty}
%     P_{ii}^n = \infty \).
% \end{theorem}
% The proof if this theorem is not immediate, see
% \cite[pages 62-67]{karlin75}.

\subsection*{Absorbing States} Up to this point the only example has
been an irreducible, positive recurrent and aperiodic Markov chain.
Now consider the following transition matrix:
\[
    P =
    \begin{pmatrix}
        1 & 0 & 0 & 0 & 0 \\
        1/2 & 0 & 1/2 & 0 & 0 \\
        0 & 1/2 & 0 & 1/2 & 0 \\
        0 & 0& 1/2 & 0 & 1/2 \\
        0 & 0 & 0 & 0 & 1
    \end{pmatrix}
    .
\]

Note that if the process starts in state \( x_1 \) or \( x_5 \), it
can't leave these states.  Similarly, if the process starts in any other
state and ``arrives'' in states \( x_1 \) or \( x_5 \) at time \( N \),
then it stays in those states for all steps \( n\geq N \).  Call these
\defn {absorbing states},%
\index{absorbing states}
that is, states which once entered are never left.

The transition matrix \( P \) represents the random walk on \( {x_0,x_1,x_2,x_3,x_4}
\) that stops at \( x_0 \) or \( x_4 \).  Markov chains with absorbing
states are \emph{not} irreducible.  Therefore the result above does not
apply for a limit matrix, so introduce the \defn{transition
probability matrix canonical form}.%
\index{transition probability matrix canonical form}
Given a Markov chain defined by transition matrix \( P \) with \( a \)
absorbing states and \( t \) transient states where \( a + t = k \),
then the matrix has the canonical form:
\[
    P=
    \begin{pmatrix}
        I & 0 \\
        A & T \\
    \end{pmatrix}
\] where \( I \) is the \( a\times a \) identity matrix, and \( 0 \) is
the \( a\times t \) matrix of zeros.  The random walk matrix from above
has the canonical form:
\[
    P =
    \begin{pmatrix}
        1 & 0 & 0 & 0 & 0 \\
        0 & 1 & 0 & 0 & 0\\
        1/2 & 0 & 0 & 1/2 & 0 \\
        0 & 0 & 1/2 & 0 & 1/2 \\
        0 & 1/2 & 0 & 1/2 & 0
    \end{pmatrix}
    .
\]

The matrix \( N=(I-T)^{-1} \) is called the \defn{fundamental matrix}.%
\index{fundamental matrix}
The entries of the fundamental matrix have the following interpretation,
\( N_{ij} \) is the expected number of times the chain is expected to be
at state \( x_j \) before absorption, given it started in state \( x_i \).
So in the random walk example:
\[
    N = (I-T)^{-1} =
    \begin{pmatrix}
        3/2 & 1 & 1/2 \\
        1 & 2 & 1\\
        1/2 & 1 & 3/2
    \end{pmatrix}
    .
\]

Letting \( \mathbf{1} \) be the \( 1 \times t \) vector of all ones, \( t= \mathbf {1} N\)
is the expected number of steps for a transient state to be absorbed in one of the
absorbing states, in this case \( (3,4,3) \). Define \( B=NA \), so \( B_{ij} \) is the probability
that starting in the \( i \)th transient state, the process ends up
in the \( j \)th absorbing state.  In the example:
\[
    B = (I-T)^{-1}A =
    \begin{pmatrix}
        3/4 & 1/4 \\
        1/2 & 1/2 \\
        1/4 & 3/4
    \end{pmatrix}
    .
\]

The canonical form gives information  about \( P^n \) for large values of \( n \).
As \( n\to\infty \), \( P^n \) approaches the matrix\( P^\infty \)
defined as:
\[
    P^\infty =
    \begin{pmatrix}
        I & 0 \\
        B & 0 \\
    \end{pmatrix}
    .
\]

\hr

\visual{Section Starter Question}{../../../../CommonInformation/Lessons/question_mark.png}
\section*{Section Ending Answer} Use the Law of Total Probability to
calculate this probability.  This considers every possible weather
combination with their respective conditional probabilities.
\[
    \Prob{O_2 = N \given O_0 = R} = \frac{1}{2} \cdot \frac{1}{4} +
    \frac{1}{4} \cdot 0 + \frac{1}{4} \cdot \frac{1}{4} = \frac{3}{16}.
\]

\subsection*{Sources}

This section is adapted from:  Notes prepared by LT Grant and used with
permission.  Pieces of this section are adapted from:  \booktitle{Finite
Markov Chains}, Kemeny and Snell, 1960, and from \booktitle{Random Walks
and Electric Networks}, Doyle and Snell, 1984, as well as \booktitle{A
First Course in Stochastic Processes} by S. Karlin and H. Taylor,
Chapter 2, 1975.

\hr

\visual{Algorithms, Scripts, Simulations}{../../../../CommonInformation/Lessons/computer.png}
\section*{Algorithms, Scripts, Simulations}

\subsection*{Algorithm}

\begin{codebox}
    \Procname{Markov Chain Simulation} \zi Comment Post:  
    Information about a simple Markov chain \li Load Markov chain
    library \li Set state names \li Set transition probability matrix
    \li Set start state \li Set up Markov chain with the library \li Set
    an example length \li Create a sample path of example length \li
    Create a second sample path of example length \li Set a long path
    length, and a transient time \li Create a long sample path \li Slice
    the long sample path from the transient time to the end \li In the
    slice count the appearance of each state \li Store in an empirical
    array \li Compute the theoretical stable array \li \Return Empirical
    stable distribution \li \Return Theoretical stable distribution
\end{codebox}

\subsection*{Scripts}

\input{markovchain_scripts}

\hr

\visual{Problems to Work}{../../../../CommonInformation/Lessons/solveproblems.png}
\section*{Problems to Work for Understanding}
\renewcommand{\theexerciseseries}{}
\renewcommand{\theexercise}{\arabic{exercise}}

\begin{exercise}
    Students' progress through a (two-year) Associate Degree program is
    not always a two year process.  A student taking first year courses
    has a \( 15\% \) chance of dropping out, a \( 25\% \) chance of
    repeating the first year, and a \( 60\% \) chance of moving on to
    the second year.  A student taking second year material has a \( 10\%
    \) chance of dropping out, a \( 20\% \) of repeating the second
    year, and a \( 70\% \) chance of graduating.
    \begin{enumerate}[label=(\alph*)]
    \item
        How many states are in this model?
    \item
        Which states are absorbing?  Which are transient?
    \item
        What is the chance a student entering the program this year will
        graduate ``on time''?  What is the chance a student will
        graduate in 3 years?
    \item
        What is the chance that a student will graduate eventually?
        What is the chance the student will drop out?
    \item
        What is the expected time until a student graduates?
\end{enumerate}
\end{exercise}
\begin{solution}
    Problem created by LT Grant, 2006.
    \begin{enumerate}[label=(\alph*)]
    \item
        There are \( 4 \) states, \( D \) for dropping out, \( 1 \) for
        first-year, \( 2 \) for second-year, and \( G \) for graduating.
        \[
            P = \bordermatrix{ & D & 1 & 2 & G \cr
            D& 1 & 0 & 0 & 0 \cr
            1& 3/20 & 1/4 & 3/5 & 0 \cr
            2& 1/10 & 0 & 1/5 & 7/10 \cr
            G& 0 & 0 & 0 & 1 }.
        \]
    \item
        Absorbing states are \( D \) and \( G \). Transient states are \(
        1 \), \( 2 \).
    \item
        \[
            P^2 =
            \begin{pmatrix}
                1 & 0 & 0 & 0 \\
                99/400 & 1/16 & 27/100 & 21/50 \\
                31/250 & 0 & 1/125 & 217/250 \\
                0 & 0 & 0 & 1 \\
            \end{pmatrix}
            .
        \] Probability of graduating on time is \( 21/50 \).
        \[
            P^3 =
            \begin{pmatrix}
                1 & 0 & 0 & 0 \\
                2271/8000 & 1/64 & 183/2000 & 609/1000 \\
                31/250 & 0 & 1/125 & 217/250 \\
                0 & 0 & 0 & 1 \\
            \end{pmatrix}
            .
        \]
    \item
        Probability of graduating in \( 3 \) years is \( 609/1000 \).
    \item
        Canonical form is
        \[
            \begin{pmatrix}
                1 & 0 & 0 & 0 \\
                0 & 1 & 0 & 0 \\
                3/20 & 0 & 1/4 & 3/5 \\
                1/10 & 7/10& 0 & 1/5
            \end{pmatrix}
        \] so
        \[
            N =
            \begin{pmatrix}
                4/3 & 1 \\
                0 & 7/8 \\
            \end{pmatrix}
            \qquad B = NA =
            \begin{pmatrix}
                3/10 & 7/10 \\
                1/8 & 7/8
            \end{pmatrix}
            .
        \] The chance that a student will graduate eventually is \( 7/10
        \).  The chance the student will drop out eventually is \( 3/10 \).
    \item
        The expected time to graduation is \( 4/3 + 1 = 7/3 \) years.
\end{enumerate}
\end{solution}
\begin{exercise}
    Consider the grid in Figure~%
    \ref{fig:markovchains:grid} upon which a random walker is stuck. If
    she reaches states \( E \) and \( F \) she will stop, otherwise she
    will keep walking.  At any of the other states, she is equally
    likely to choose any of the possible neighboring states.
    \begin{figure}
        \centering
\begin{asy}
                size(2inches);

                real myfontsize = 12; real mylineskip = 1.2*myfontsize;
                pen mypen = fontsize(myfontsize, mylineskip); defaultpen
                (mypen);

                dot("\( E \)", (0,2), N); dot("\( A \)", (0,1), W); dot("\(
                C \)", (0,0), S); dot("\( B \)", (1,1), E); dot("\( D \)",
                (1,0), S); dot("\( F \)", (2,0), E);

                draw((0,2)--(0,0)--(2,0)); draw((0,1)--(1,1)--(1,0));
\end{asy}
        \caption{Grid for the random walk.}%
        \label{fig:markovchains:grid}
    \end{figure}
    \begin{enumerate}[label=(\alph*)]
    \item
        What is the transition matrix?
    \item
        What are the absorbing states?  The transient states?
    \item
        What is the canonical form of this matrix?  The fundamental
        form?
    \item
        Starting at state \( A \), on average, how many steps will the
        walker take before reaching states \( E \) or \( F \)?
    \item
        Starting at state \( A \), what is the probability the walker
        will end at the state \( F \)?  Starting at \( D \) what is the
        probability the walker ends at state \( E \)?  Discuss these two
        answers.
\end{enumerate}
\end{exercise}
\begin{solution}
    Problem created by LT Grant, 2006.
    \begin{enumerate}[label=(\alph*)]
    \item
        \[
            P = \bordermatrix{ & A & B & C & D & E & F \cr
            A & 0 & 1/3 & 1/3 & 0 & 1/3 & 0 \cr
            B & 1/2 & 0 & 0 & 1/2 & 0 & 0 \cr
            C & 1/2 & 0 & 0 & 1/2 & 0 & 0 \cr
            D & 0 & 1/3 & 1/3 & 0 & 0 & 1/3 \cr
            E & 0 & 0 & 0 & 0 & 1 & 0 \cr
            F & 0 & 0 & 0 & 0 & 0 & 1 }.
        \]
    \item
        Absorbing states are \( E \) and \( F \). Transient states are \(
        A \), \( B \), \( C \), \( D \).
    \item
        \[
            P' = \bordermatrix{ & E & F & A & B & C & D \cr
            E & 1 & 0 & 0 & 0 & 0 & 0 \cr
            F & 0 & 1 & 0 & 0 & 0 & 0 \cr
            A & 1/3 & 0 & 0 & 1/3 & 1/3 & 0 \cr
            B & 0 & 0 & 1/2 & 0 & 0 & 1/2 \cr
            C & 0 & 0 & 1/2 & 0 & 0 & 1/2 \cr
            D & 0 & 1/3 & 0 & 1/3 & 1/3 & 0 \cr
            }.
        \]
        \[
            N =
            \begin{pmatrix}
                2 & 1 & 1 & 1\\
                \frac{3}{2} & 2 & 1 & \frac{3}{2}\\
                \frac{3}{2} & 1 & 2 & \frac{3}{2}\\
                1 & 1 & 1 & 2
            \end{pmatrix}
            .
        \]
    \item
        \( (N \mathbf{1})_{1} = 2 + 1 + 1 + 1 = 5 \).
    \item
        From \( A \), probability to \( E \) is \( 2/3 \), to \( F \) is
        \( 1/3 \). From \( D \), probability to \( E \) is \( 1/3 \), to
        F is \( 2/3 \). Seems reasonable by symmetry.
\end{enumerate}
\end{solution}
\begin{exercise}
    The walker returns a year later to the same grid, only to find that
    there now exists a road between \( B \) and \( E \).
    \begin{enumerate}[label=(\alph*)]
    \item
        Without any calculations, how would you expect this to change
        the absorption probabilities of state \( B \)?
    \item
        Without any calculations, would you expect the same results as
        you found in part (e) above?
    \item
        Find the new transition matrix.
    \item
        Calculate the actual probabilities for part (a) and (b) and
        comment on your results.
\end{enumerate}
\end{exercise}
\begin{solution}
    Problem created by LT Grant, 2006.
    \begin{enumerate}[label=(\alph*)]
    \item
        No, expect absorption probabilities to increase.
    \item
        From \( A \), probability to \( E \) to increase, to \( F \) to
        decrease. From \( D \), probability to \( E \) to increase, to \(
        F \) to decrease.
    \item
        \[
            \begin{pmatrix}
                0 & \frac{1}{3} & \frac{1}{3} & 0 & \frac{1}{3} & 0\\
                \frac{1}{3} & 0 & 0 & \frac{1}{3} & \frac{1}{3} & 0\\
                \frac{1}{2} & 0 & 0 & \frac{1}{2} & 0 & 0\\
                0 & \frac{1}{3} & \frac{1}{3} & 0 & 0 & \frac{1}{3}\\
                0 & 0 & 0 & 0 & 1 & 0\\
                0 & 0 & 0 & 0 & 0 & 1
            \end{pmatrix}
            .
        \]
    \item
        From \( A \), probability to \( E \) is \( 0.792 \), to \( F \)
        to decrease is \( 0.208 \). From \( D \), probability to \( E \)
        is approximately \( 0.458 \), to \( F \) is approximately \(
        0.542 \).
\end{enumerate}
\end{solution}
\begin{exercise}
    Determine the transition probability matrix for the following Markov
    Chain:  \( N \) black balls and \( N \) white balls are placed in
    two urns so that each contains \( N \) balls.  At each step one ball
    is selected at random from each urn and the two balls interchange
    urns.  The state of the system is the number of white balls in the
    first urn.  What are the absorbing states, transient states,
    recurrent states, if any?  Is this Markov chain ergodic?
\end{exercise}
\begin{solution}
    Adapted from Karlin and Taylor, \booktitle{A First Course in
    Stochastic Processes}, second edition, page 73, problem 1(b).
    % \begin{align*}
    %   P_{jk} &= \Prob{ k \text{ white balls in first urn after $n+1$
    %   interchanges} \given \\
    %          & \qquad j \text{ white balls in first urn after $n$
    %            interchanges}}
    %      \end{align*}
    \[
        P_{jk} =
        \begin{cases}
            (\frac{j}{N})^2 \qquad k = j-1, \qquad j=1,2,\dots,N \\
            (2\frac{j}{N})\frac{N-j}{N} \qquad k = j, \qquad j=0,1,2,\dots,N
            \\
            1- (\frac{j}{N})^2 \qquad k = j-1, \qquad j=0,1,2,\dots,N-1
            \\
        \end{cases}
    \] There are no absorbing states, all states are recurrent.  Yes
    ergodic. As an example, for \( N = 3 \), the transition probability
    matrix is
    \[
        \bordermatrix{ & 0 & 1 & 2 & 3 \cr
        0 & 0 & 1 & 0 & 0 \cr
        1 & 1/9 & 4/9 & 4/9 & 0 \cr
        2 & 0 & 4/9 & 4/9 & 1/9 \cr
        3 & 0 & 0 & 1 & 0 }
    \] with stationary distribution \( \pi = (1/20, 9/20, 9/20, 1/20) \)
    For \( N = 4 \), the transition probability matrix is
    \[
        \bordermatrix{ & 0 & 1 & 2 & 3 & 4 \cr
        0 & 0 & 1 & 0 & 0 & 0 \cr
        1 & 1/16& 6/16& 9/16& 0 & 0 \cr
        2 & 0 & 4/16& 8/16& 4/16 & 0 \cr
        3 & 0 & 0 & 9/26& 6/16 & 1/16 \cr
        4 & 0 & 0 & 0 & 1 & 0 }
    \]
\end{solution}
\begin{exercise}
    Determine the transition probability matrix for the following Markov
    Chain:  Consider two urns A and B with a total of \( N \) balls.  An
    experiment is performed in which a ball is selected at random (all
    selections equally likely) at time \( n \), \( n=1,2,\dots \) from
    among the totality of \( N \) balls.  Then an urn is selected at
    random, urn A with probability \( p \) and urn B with probability \(
    q = 1- p \) and the ball previously drawn is placed in this urn.
    The state of the system is the number of balls in the urn A. What
    are the absorbing states, transient states, recurrent states, if
    any?  Is this Markov chain ergodic?
\end{exercise}
\begin{solution}
    Adapted from Karlin and Taylor, \booktitle{A First Course in
    Stochastic Processes}, second edition, page 74, problem 2(a).
    % P_{jk} = \Pr[ k \text{ balls in first urn after $n+1$
    % interchanges} | j \text{ balls in first urn after $n$
    % interchanges}]
    \[
        P_{jk}
        \begin{cases}
            \left( \frac{j}{N}\right)q \qquad k = j-1, \qquad j=1,2,\dots,N
            \\
            \left( \frac{j}{N} \right) p + \left( \frac{N-j}{N} \right)
            q \qquad k = j, \qquad j=0,1,2,\dots,N \\
            \left( 1- \frac{j}{N} \right) p \qquad k = j+1, \qquad
            j=0,1,2,\dots,N-1 \\
        \end{cases}
    \] An example in the case \( N=4 \)
    \[
        \begin{pmatrix}
            q & p & 0 & 0 & 0\\
            \frac{q}{4} & \frac{3q}{4}+\frac{p}{4} & \frac{3p}{4} & 0 &
            0\\
            0 & \frac{q}{2} & \frac{q}{2}+\frac{p}{2} & \frac{p}{2} & 0\\
            0 & 0 & \frac{3q}{4} & \frac{q}{4}+\frac{3p}{4} & \frac{p}{4}\\
            0 & 0 & 0 & q & p
        \end{pmatrix}
        .
    \] The unnormalized stationary distribution is \( [1, 4p/q, 6p^2/q^2,
    4p^3/q^3, p^4/q^4 ] \), for \( p = 1/2 = q \), normalized, this is \(
    [ 1/16, 1/4, 3/8, 1/4, 1/16] \), for \( p =3/4 \), \( q = 1/4 \),
    normalized, this is approximately \( [0.007, 0.082, 0.370, 0.740,
    0.555] \).
\end{solution}
\begin{exercise}
    Determine the transition probability matrix for the following Markov
    Chain:  Consider two urns A and B with a total of \( N \) balls.  An
    experiment is performed in which an urn is selected at random in
    proportion to its contents (i.e.\ if urn A has exactly \( k \)
    balls, then it is selected with probability \( k/N \). The a ball is
    selected from A with probability \( p \) and from urn B with
    probability \( q = 1-p \) and placed in the previously chosen urn.
    The state of the system is the number of balls in the urn A. What
    are the absorbing states, transient states, recurrent states, if
    any?  Is this Markov chain ergodic?
\end{exercise}
\begin{solution}
    Adapted from Karlin and Taylor, \booktitle{A First Course in
    Stochastic Processes}, second edition, page 74, problem 2(b).
    % \Pr[ k \text{ white balls in first urn after $n+1$
    %   interchanges} | j \text{white balls in first urn after $n$
    %   interchanges}]
    %   =
    \[
        P_{jk} =
        \begin{cases}
            \left( 1-\frac{j}{N}\right)p \qquad k = j-1, \qquad j=1,2,\dots,N
            \\
            \left( \frac{j}{N} \right) p + \left( \frac{N-j}{N} \right)
            q \qquad k = j, \qquad j=0,1,2,\dots,N \\
            \left( \frac{j}{N} \right) q \qquad k = j+1, \qquad j=0,1,2,\dots,N-1.
            \\
        \end{cases}
    \]
\end{solution}
\begin{exercise}
    Your baby is learning to walk.  The baby begins by holding onto a
    couch.  Whenever she is next to the couch, there is a \( 25 \)
    percent chance that she will take a step forward and a \( 75 \)
    percent chance that she will stay clutching the couch.  If the baby
    is one or more steps away from the couch, there’s a \( 25 \)
    percent chance that she will take a step forward, a \( 25 \) percent
    chance she’ll stay in place and a \( 50 \) percent chance she’ll
    take one step back toward the couch.  In the long run, what percent
    of the time does the baby choose to clutch the couch?
\end{exercise}
\begin{solution}
    Adapted from \link{https://fivethirtyeight.com/features/will-the-baby-walk-away-will-the-troll-kill-the-dwarves/}
    {FiveThirtyEight Riddler} who in turn got the problem from Steve
    Simon. From numerical experiments with the transition probability
    matrix, about \( 50\% \) of the time.  This matrix converges fairly
    slowly to the limiting matrix, so experiments require high powers.
\end{solution}
\begin{exercise}
    Louie lives in a town that had a \( 50 \) percent chance of rain
    each morning and an independent \( 40 \) percent chance each
    evening.  Louie walks to and from work each day, bringing an
    umbrella with him if it was raining and not bringing one if it
    wasn't.  On Sunday night, two of his three umbrellas were with him
    at home and one was at his office.  What were the chances that he
    made it through the work week without getting wet?
\end{exercise}
\begin{solution}
    Adapted from \link{https://fivethirtyeight.com/features/the-little-mathematically-determined-house-on-the-prairie/}
    {FiveThirtEight Riddler} who in turn got the problem from Josh
    Vandenham.)
    \begin{remark}
        q This problem and the solution as a Markov Chain appeared in
        \link{https://fivethirtyeight.com/features/can-you-win-tic-tac-toe-blindfolded/}
        {FiveThirtyEight Riddler column of December 14, 2018.}
    \end{remark}

    Just keep track of the different ``states'' of Louie’s umbrellas.
    He could have three at home and zero at work, two at home and one at
    work, one at home and two at work, or zero at home and three at
    work.  The movement through the random states (the work week) is
    described by the transition matrix, which contains the probabilities
    of umbrellas moving from one place to another on any given day.  In
    our particular case, the matrix looks like this:
    \[
        \begin{pmatrix}
            0.3 & 0.2 & 0.0 & 0.0 & 0.5\\
            0.3 & 0.5 & 0.2 & 0.0 & 0.0\\
            0.0 & 0.3 & 0.5 & 0.2 & 0.0\\
            0.0 & 0.0 & 0.3 & 0.5 & 0.2\\
            0.0 & 0.0 & 0.0 & 0.0 & 1.0
        \end{pmatrix}
        .
    \] In this solution, there are five states:  zero, one, two or three
    umbrellas at home plus the state in which Louie has already gotten
    wet.  The rows of the matrix are Louie’s current state, and the
    columns are his destination.  For example, he starts in the third
    row (two umbrellas at home).  There is no chance he’ll end up with
    zero umbrellas at home, there’s a 0.3 chance he’ll end the day
    with just one at home, a 0.5 chance with two at home, a 0.2 chance
    with three at home, and no chance he’ll get wet on this first day.
    To get the probabilities for the entire work week, simply multiply
    this matrix by itself five times, representing the meteorological
    randomness unfolding over the five days, which gives us the
    following:
    \[
        \begin{pmatrix}
            \nonumber 0.04791 &0.07634 &0.04976 &0.01776 &0.80823\\
            0.11451 &0.19889 &0.15274 &0.06752 &0.46634\\
            0.11196 &0.22911 &0.22553 &0.12610 &0.3073\\
            0.05994 &0.15192 &0.18915 &0.12425 &0.47474\\
            0 &0 &0 &0 &0
        \end{pmatrix}
        .
    \]

    Last, look at the third row (the state in which he began) and the
    final column:  That’s the chance he gets wet during the week.
    Subtract from \( 1 \) to find the chance that he didn’t get wet
    and get \( 69.27 \) percent.

    \begin{remark}
        This is a version of Ross, page 181, problem number 22
    \end{remark}
\end{solution}
\begin{exercise}
    You’re going to play a game.  Like many probability games, this
    one involves an infinite supply of ping-pong balls.

    The balls are numbered \( 1 \) through \( N \).  There is also a
    group of \( N \) cups, labeled \( 1 \) through \( N \), each of
    which can hold an unlimited number of ping-pong balls.  The game is
    played in rounds. A round is composed of two phases:  throwing and
    pruning.

    During the throwing phase, the player takes balls randomly, one at a
    time, from the infinite supply and tosses them at the cups.  The
    throwing phase is over when every cup contains at least one
    ping-pong ball.  Next comes the pruning phase.  During this phase
    the player goes through all the balls in each cup and removes any
    ball whose number does not match the containing cup.

    Every ball drawn has a uniformly random number, every ball lands in
    a uniformly random cup, and every throw lands in some cup.  The game
    is over when, after a round is completed, there are no empty cups.

    \begin{enumerate}[label=(\alph*)]
    \item
        How many balls would you expect to need to draw and throw to
        finish this game?
    \item
        For \( N = 3 \), how many rounds would you expect to need to
        play to finish this game?
\end{enumerate}
\end{exercise}
\begin{solution}
    \begin{remark}
        This problem and the solution as a Markov Chain appeared in
        \link{https://fivethirtyeight.com/features/the-riddler-just-had-to-go-and-reinvent-beer-pong/}
        {FiveThirtyEight} Riddler column of November 16, 2018.  The
        solution is adapted from \link{https://laurentlessard.com/bookproofs/beer-pong/}
        {Five Thirty Eight Riddler}.
    \end{remark}

    \begin{enumerate}[label=(\alph*)]
    \item
        For this question, the rounds don’t matter at all.  The game
        will end once there is at least one correct ball in each cup,
        and the pruning phase at the end of each round only removes
        incorrect balls so effectively neglect it.

        This is a classical coupon collector problem.  Define \( Y_i \),
        for \( i \) from \( 0 \) to \( N - 1 \), to be the number of additional
        tosses that need to be obtained after \( i \) distinct cups have
        been filled.  Define \( Y \) as the total number of tosses
        required to fill the cups with correspondingly numbered balls so
        \[
            Y = Y_0 + Y_1 + Y_2 + \cdots + Y_{N-1}.
        \] When \( i \) distinct cups have been filled with
        corresponding numbered balls, the probability that a ball with a
        number of an unfilled cup is drawn is \( \frac{N-i}{N} \).  Each
        time a ball is thrown, there is a \( 1/N \) chance it will land
        in the correct cup.  It follows that the probability a new cup
        is filled with the correctly numbered ball is \( \frac{N-i}{N}
        \cdot \frac{1}{N} \).  Thus, \( Y_i \) is a geometric random
        variable with success parameter \( \frac{N-i}{N^2} \). Then \(
        \E{Y_i} = \frac{N^2}{N-i} \).  Therefore
        \begin{align*}
            \E{Y} &= \E{Y_0} + \E{Y_1} + \E{Y_2} + \cdots + Y_{N-1} \\
            &= N + \frac{N^2}{N-1} + \frac{N^2}{N-2} + \cdots + \frac{N^2}
            {1} \\
            &= N^2 \sum_{k=1}^N \frac{1}{k} \approx N^2(\log N + \gamma)
        \end{align*}
        where \( \gamma \approx 0.5772 \) is the Euler-Mascheroni
        constant.
    \item
        This is a more complicated question, so to illustrate the
        approach solve it for the case \( N=3 \).  Begin by breaking
        each round into a number of stages, which track the progress of
        cups as they are filled.  In the diagram of a single below, each
        cup is either empty (no ball), is red (incorrect ball), or is
        green (correct ball).  The transition probabilities are on the
        arrows.

        \includegraphics[scale=0.5]{beer-pong.png}

        The integer \( k \) indicates the number of balls thrown.  The
        integer \( j \) indicates the number of cups filled holding a
        correspondingly numbered ball.  A colored cup may contain more
        than one ball.  The cup position left to right is not associated
        with the cup or ball numbers, the position merely indicates the
        number of correctly or incorrectly filled cups. For example, if \(
        2 \) balls have been thrown into \( 2 \) cups, one correctly
        numbered, one incorrectly numbered, is the center diagram.  Then
        on the next ball thrown, that ball can incorrectly go into the
        empty cup with probability \( \frac{N-i}{N} \cdot \frac{1}{N} \).
        The correct ball can go into the correct cup with probability \(
        \frac{1}{N2} \). In this diagram, self-loops have whatever
        probability is required so that the outgoing arrows sum to 1.
        This is a Markov chain.  We start on the initial state on the
        left (all cups empty), and advance to the absorbing state on the
        right (all cups full).  The transition matrix for \( N=3 \),
        arranging the states from top to bottom and left to right, is:
        \[
            A =
            \begin{pmatrix}
                0 & \frac{2}{3} & \frac{1}{3} & 0 & 0 & 0 & 0 & 0 & 0 &
                0\\
                0 & \frac{2}{9} & \frac{1}{9} & \frac{4}{9} & \frac{2}{9}
                & 0 & 0 & 0 & 0 & 0\\
                0 & 0 & \frac{1}{3} & 0 & \frac{4}{9} & \frac{2}{9} & 0
                & 0 & 0 & 0\\
                0 & 0 & 0 & \frac{4}{9} & \frac{2}{9} & 0 & \frac{2}{9}
                & \frac{1}{9} & 0 & 0\\
                0 & 0 & 0 & 0 & \frac{5}{9} & \frac{1}{9} & 0 & \frac{2}
                {9} & \frac{1}{9} & 0\\
                0 & 0 & 0 & 0 & 0 & \frac{2}{3} & 0 & 0 & \frac{2}{9} &
                \frac{1}{9}\\
                0 & 0 & 0 & 0 & 0 & 0 & 1 & 0 & 0 & 0\\
                0 & 0 & 0 & 0 & 0 & 0 & 0 & 1 & 0 & 0\\
                0 & 0 & 0 & 0 & 0 & 0 & 0 & 0 & 1 & 0\\
                0 & 0 & 0 & 0 & 0 & 0 & 0 & 0 & 0 & 1
            \end{pmatrix}
            .
        \]

        First, eliminate the self-loops by normalizing because it
        doesn't matter how much time is spent at each stage, the state
        of the Markov chain hasn't changed.  See below for an
        illustration of this normalization, based on the observation
        that \( \Prob{B \given ~A} = \Prob{B \given B \union C} = \frac{p}
        {p+q} \).

        \includegraphics[scale=0.9]{normalization.png}

        Here is what our transition matrix looks like after
        normalization.
        \[
            A_{\text{norm}} =
            \begin{pmatrix}
                0 & \frac{2}{3} & \frac{1}{3} & 0 & 0 & 0 & 0 & 0 & 0 &
                0\\
                0 & 0 & \frac{1}{7} & \frac{4}{7} & \frac{2}{7} & 0 & 0
                & 0 & 0 & 0\\
                0 & 0 & 0 & 0 & \frac{2}{3} & \frac{1}{3} & 0 & 0 & 0 &
                0\\
                0 & 0 & 0 & 0 & \frac{2}{5} & 0 & \frac{2}{5} & \frac{1}
                {5} & 0 & 0\\
                0 & 0 & 0 & 0 & 0 & \frac{1}{4} & 0 & \frac{1}{2} &
                \frac{1}{4} & 0\\
                0 & 0 & 0 & 0 & 0 & 0 & 0 & 0 & \frac{2}{3} & \frac{1}{3}\\
                0 & 0 & 0 & 0 & 0 & 0 & 1 & 0 & 0 & 0\\
                0 & 0 & 0 & 0 & 0 & 0 & 0 & 1 & 0 & 0\\
                0 & 0 & 0 & 0 & 0 & 0 & 0 & 0 & 1 & 0\\
                0 & 0 & 0 & 0 & 0 & 0 & 0 & 0 & 0 & 1
            \end{pmatrix}
            .
        \]

        Once again, the rows sum to \( 1 \), but this time only the
        absorbing states have self-loops.  Putting the matrix into
        canonical form:
        \[
            A_{\text{canon}} =
            \begin{pmatrix}
                1 & 0 & 0 & 0 & 0 & 0 & 0 & 0 & 0 & 0\\
                0 & 1 & 0 & 0 & 0 & 0 & 0 & 0 & 0 & 0\\
                0 & 0 & 1 & 0 & 0 & 0 & 0 & 0 & 0 & 0\\
                0 & 0 & 0 & 1 & 0 & 0 & 0 & 0 & 0 & 0\\
                0 & 0 & 0 & 0 & 0 & \frac{2}{3} & \frac{1}{3} & 0 & 0\\
                0 & 0 & 0 & 0 & 0 & 0 & \frac{1}{7} & \frac{4}{7} &
                \frac{2}{7} & 0\\
                0 & 0 & 0 & 0 & 0 & 0 & 1 & \frac{2}{3} & \frac{1}{3} &
                0\\
                \frac{2}{5} & \frac{1}{5} & 0 & 0 & 0 & 0 & 0 & 0 &
                \frac{2}{5} & 0\\
                0 & \frac{1}{2} & \frac{1}{4} & 0 & 0 & 0 & 0 & 0 & 0 &
                \frac{1}{4} \\
                0 & 0 & \frac{2}{3} & \frac{1}{3} & 0 & 0 & 0 & 0 & 0 &
                0
            \end{pmatrix}
            .
        \] The normal matrix \( (I = T)^{-1} \) is then
        \[
            \begin{pmatrix}
                1 & \frac{2}{3} & \frac{3}{7} & \frac{2}{3} & \frac{7}{9}
                & \frac{7}{36}\\
                0 & 1 & \frac{1}{7} & \frac{2}{3} & \frac{7}{9} & \frac{7}
                {36}\\
                0 & 0 & 1 & \frac{2}{3} & \frac{7}{9} & \frac{7}{36}\\
                0 & 0 & 0 & 1 & \frac{2}{3} & \frac{1}{6}\\
                0 & 0 & 0 & 0 & 1 & \frac{1}{4}\\
                0 & 0 & 0 & 0 & 0 & 1
            \end{pmatrix}
            .
        \] Then the expected number of throws until the cups are filled
        is (as a column vector)
        \[
            \left( \frac{941}{252}, \frac{701}{252}, \frac{95}{36},
            \frac{11}{6}, \frac{5}{4}, 1 \right)^{T}.
        \]

        % Finding the limiting
        % distribution simply amounts to finding the limit \( \lim_{n \to
        % \infty}A_{\text{norm}}^{n} \).  Since the upper-left block of \(
        % A_{\text{norm}} \) is nilpotent (the diagonal is zero), powers
        % of the matrix will eventually be constant.  It suffices to
        % evaluate \( A^{2N}_{\text{norm}} \).  Extracting the last \( N+1
        % \) rows and the columns corresponding to the states \( \{ 1,3,6,
        % \dots,(N+1)(N+2)/2\} \) (the states where we have exactly \( k \)
        % correct balls in cups, for \( k=0,1,\dots,N \)), we obtain the
        % transition matrix for one round of the game.  Here is the result
        % we obtain in the case \( N=3 \):
        % \[
        %     P =
        %     \begin{pmatrix}
        %         16/105 & 41/105 & 5/14 & 1/10 \\
        %         0 & 1/3 & 0 & 0 \\
        %         0 & 0 & 2/3 & 1/3 \\
        %         0 & 0 & 0 & 1
        %     \end{pmatrix}
        % \] In other words, \( P_{ij} \) is the probability that if we
        % currently have \( i \) correct balls in cups, we will have \( j \)
        % after the next round is over.  From here, our next task is to
        % compute the expected number of rounds it will take to travel
        % from \( k=0 \) to \( k=N \).  This can be computed as follows:
        % if we call \( E_k \) the expected number of rounds starting from
        % \( k \), then we have:
        % \begin{align*}
        %     E_0 &= 1 + P_{00} E_0 + P_{10}E_1 + \cdots + P_{N0}E_N \\
        %     E_1 &= 1 + P_{01} E_0 + P_{11}E_1 + \cdots + P_{N1}E_N \\
        %     &\vdots \\
        %     E_{N-1} &= 1 + P_{0,N-1} E_0 + P_{1,N-1}E_1 + \cdots + P_{N,N-1}
        %     E_N \\
        %     E_N &= 0
        % \end{align*}

        % In matrix-vector notation:  \( E=1–\mathbf{e}_{N}+P^{T}E \).
        % This can be solved via simple matrix inversion.  We can then
        % extract \( E_0 \) from the solution, which is the expected
        % number of rounds starting with all empty cups.  There does not
        % appear to be an easy way to extract an analytic formula, but all
        % the steps above are easy to compute numerically.
\end{enumerate}

\end{solution}

\hr

\visual{Books}{../../../../CommonInformation/Lessons/books.png}
\section*{Books}

\bibliography{../../../../CommonInformation/bibliography}

% \begin{enumerate}
%     \item
%         \booktitle{A First Course in Stochastic Processes} second edition,
%         S. Karlin and H. Taylor, Academic Press, 1975.  Chapter 2.
%     \item \booktitle{Finite Markov Chains} by John G. Kemeny and
%       J. Laurie Snell, D. Van Nostrand, 1960.
%     \item \booktitle{Introduction to Probability Models}, Sheldon
%       M. Ross, Elsevier. 1997, Chapter 4.
% \end{enumerate}

\hr

\visual{Links}{../../../../CommonInformation/Lessons/chainlink.png}
\section*{Links}
\begin{enumerate}
    \item
        \booktitle{Introduction to Probability}, Grinstead and Snell.
        Chapter 11 - Markov Chains http://tinyurl.com/qw6sa \link{http://www.dartmouth.edu/~chance/teaching_aids/books_articles/probability_book/Chapter11.pdf}
        {Introduction to Probability, Chapter 11} Accessed Wed Jan 16
        07:32:50 CST 2019

    \item
        \link{https://cran.r-project.org/web/packages/markovchain/vignettes/an_introduction_to_markovchain_package.pdf}
        {Markov The markovchain package} Accessed Wed Jan 16 07:34:15
        CST 2019

    \item
        \link{http://www.dartmouth.edu/~chance/teaching_aids/books_articles/probability_book/Chapter11.pdf}
        {Getting Started with Markov Chains} Accessed Wed Jan 16
        07:32:22 CST 2019
\end{enumerate}

\section*{\solutionsname} \loadSolutions

\hr

\mydisclaim \myfooter

Last modified:  \flastmod

\end{document}

File name                  : markovchain.tex
Number of characters       : 58420
Number of words            : 6433
Percent of complex words   : 14.77
Average syllables per word : 1.6806
Number of sentences        : 332
Average words per sentence : 19.3765
Number of text lines       : 1229
Number of blank lines      : 212
Number of paragraphs       : 188


READABILITY INDICES

Fog                        : 13.6576
Flesch                     : 44.9930
Flesch-Kincaid             : 11.7974


